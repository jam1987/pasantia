 \addcontentsline{toc}{chapter}{Introducci\'on}

% Titulo de la introduccion.
\begin{center}
	{\bf Introducci\'on} \label{chap:intro}
\end{center}

% Contenido de la introduccion.

\label{sect:motivacion}


AQU\'I VA EL CONTENIDO DE ANTECEDENTES. 



\label{sect:justificacion}

\label{sect:planteamiento}
%Puedes quitar esto(es opcional)
\vspace{5 mm}

\vspace{5 mm}

\label{sect:objetivo_general}
%Puedes quitar esto(es opcional)
\vspace{5 mm}

\indent El objetivo principal del proyecto es la de la implementaci'on de una soluci'on \emph{SAP SD} en una empresa de bebidas de consumo masivo \emph{SSA CPG} para gestionar los procedimientos comerciales. 


%Puedes quitar esto(es opcional)
\vspace{5 mm}

\label{sect:objetivos_especificos}
%Puedes quitar esto(es opcional)
\vspace{5 mm}

 Los objetivos espe'ificos planteados en este proyecto son los que se listan a continuaci'on:
\begin{enumerate}
\item Adquirir los conocimientos de la Metodología Ascendant SAP y el funcionamiento del Sistema SAP en la funcionalidad Ventas y Distribuci'on (SD).

\item Analizar los procesos de Ventas y Distribuci'on con las funcionalidades SAP que los soportan. Adquirir los conocimientos generales del negocio, procesos y visi'on global de la soluci'on SAP a implantar.

\item Configurar los campos del sistema a implantar para el m'odulo SD acorde a las necesidades del cliente, hacer los desarrollos necesarios y probar las funcionalidades SAP ERP SD que soportan los procesos comerciales.

\item Preparar los datos y el sistema, realizar la demostraci'on y documentar los resultados correspondientes al m'odulo SD, para comprobar el funcionamiento y configuraci'on del m'odulo de manera individual y en relaci'on con otros m'odulos implicados del sistema.

\end{enumerate}




