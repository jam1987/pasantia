 \addcontentsline{toc}{chapter}{Introducci\'on}

% Titulo de la introduccion.
\begin{center}
	{\bf Introducci\'on} \label{chap:intro}
\end{center}

% Contenido de la introduccion.

\label{sect:motivacion}
%Puedes quitar esto(es opcional)
\vspace{5 mm}

AQU\'I VA EL CONTENIDO DE ANTECEDENTES. 

%Puedes quitar esto(es opcional)
\vspace{5 mm}

\label{sect:justificacion}
%Puedes quitar esto(es opcional)
\vspace{5 mm}

AQU\'I VA EL CONTENIDO DE LA JUSTIFICACI\'ON.
%Puedes quitar esto(es opcional)
\vspace{5 mm}

En la Universidad de Tohoku de Sendai, Jap'on, en el laboratorio Kinoshita de
la facultad de Ciencias de la Computaci'on, se tiene un {\it framework} con el
cual han creado un {\it Active Information Resource} (AIR) - {\it Network
Management System} (NMS) \cite{kinoshitaNMSAIR} para gestionar las redes de
computadoras. Mediante el lenguaje de programaci'on orientado a agentes
basado en Java, DASH ({\it Repository-based Agent Framework}) y su ambiente
de dise~no interactivo IDEA ({\it Interactive Design Environment}), la idea
final es que la red y los elementos que la componen se pueda auto-gestionar
mediante los agentes instalados en los elementos \cite{545112}, sin la
necesidad de tener un administrador. El objectivo es identificar y solucionar
los problemas de la red en tiempo real \cite{AbarAK04} \cite{KonnoAIK07}.

%Puedes quitar esto(es opcional)
\vspace{5 mm}


\label{sect:planteamiento}
%Puedes quitar esto(es opcional)
\vspace{5 mm}

AQU\'I VA EL CONTENIDO DEL PLANTEAMIENTO DEL PROBLEMA.

En los 'ultimos a~nos ha habido un gran inter'es por determinar si los nuevos
modelos basados en la autosimilaridad y la dependencia de largo alcance pueden
ser utilizados para la detecci'on de anomal'ias en las redes de datos
\cite{5475821}. Entre las anomal'ias m'as destacadas se encuentran las de
ataques de denegaci'on de servicio y su ampliaci'on: el llamado ataque
distribuido de denegaci'on de servicio \cite{mingliddos} \cite{xiang:292}.

En el trabajo de \cite{mingliddos} se utiliz'o, con buenos resultados, el
c'alculo del grado de autosimilaridad para detectar anomal'ias en la red. Se
realiz'o una implementaci'on para un {\it framework} de autogesti'on de redes
desarrollado en el laboratorio Kinoshita de la Universidad de Tohoku. Como se
explica en la pr'oxima secci'on la idea del proyecto de grado fue mejorar esta
implementaci'on.

%Puedes quitar esto(es opcional)
\vspace{5 mm}

\label{sect:objetivo_general}
%Puedes quitar esto(es opcional)
\vspace{5 mm}

\indent El objetivo principal del proyecto es la de la implementaci'on de una soluci'on \emph{SAP SD} en una empresa de bebidas de consumo masivo \emph{SSA CPG} para gestionar los procedimientos comerciales. 


%Puedes quitar esto(es opcional)
\vspace{5 mm}

\label{sect:objetivos_especificos}
%Puedes quitar esto(es opcional)
\vspace{5 mm}

 Los objetivos espe'ificos planteados en este proyecto son los que se listan a continuaci'on:
\begin{enumerate}
\item Adquirir los conocimientos de la Metodología Ascendant SAP y el funcionamiento del Sistema SAP en la funcionalidad Ventas y Distribuci'on (SD).

\item Analizar los procesos de Ventas y Distribuci'on con las funcionalidades SAP que los soportan. Adquirir los conocimientos generales del negocio, procesos y visi'on global de la soluci'on SAP a implantar.

\item Configurar los campos del sistema a implantar para el m'odulo SD acorde a las necesidades del cliente, hacer los desarrollos necesarios y probar las funcionalidades SAP ERP SD que soportan los procesos comerciales.

\item Preparar los datos y el sistema, realizar la demostraci'on y documentar los resultados correspondientes al m'odulo SD, para comprobar el funcionamiento y configuraci'on del m'odulo de manera individual y en relaci'on con otros m'odulos implicados del sistema.

\end{enumerate}




