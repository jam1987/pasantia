% Conclusiones
\chapter{Conclusiones y recomendaciones} \label{chap:conclusiones}

	Las actividades que fueron programadas para la configuraci'on del M'odulo SD de SAP para \textit{SSA Beverage} fueron realizadas con 'exito. Para ello, se llevaron a cabo cada una de las fases de la metodolog'ia \textbf{Ascendant SAP} con el fin de que el proyecto pudiera llegar a feliz t'ermino.  En las primeras dos fases se captaron los requerimientos necesarios de la empresa para el nuevo sistema a implantar, como por ejemplo toda la parte organizativa de la misma, los aspectos relacionados con el proceso de Ventas y Distribuci'on que 'esta lleva a cabo (Tipo de Pedidos, Tipos de Entregas, Tipos de Factura). 
	La informaci'on recaudada permiti'o que en las fases sucesivas se pudiera llevar a cabo la configuraci'on del m'odulo de Ventas y Distribuci'on. Para ello se procedi'o a ingresar la informaci'on relacionada con la parte estructural de la empresa, como por ejemplo: las organizaciones de ventas, sectores, canales de distribuci'on, etc. Adicionalmente, se realiz'o la configuraci'on del manejo de ventas de la empresa. Para esto, se cre'o un tipo de pedido, de entrega y de factura. Junto con esto, se cre'o el esquema de precio que maneja la empresa para la venta de sus productos. En el mismo se detallan los precios brutos, netos e impuestos a aplicar. Para la impresi'on de la factura fue necesario la creaci'on de dos elementos importantes: La clase de mensaje y un formulario. La primera, fue para que 'esta tuviera la informaci'on de la rutina que construye el formulario. El segundo, consisti'o en el dise\~no visual de la factura, tomando en cuenta las legislaciones vigentes por el Seniat.
	Aunado a las configuraciones, se desarroll'o el programa de Carga Masiva de Clientes en el lenguaje de programaci'on \textbf{ABAP/4}, con el cual \textit{SSA Beverage} podr'a cargar de una manera m'as r'apida y sencilla su cartera de clientes a trav'es de un archivo en formato \textbf{.xls}. De esta manera tendr'ia dos opciones para cargar datos en el maestro de clientes: de a uno por la transacci'on \textbf{XD01} de SAP, o en lote por la transacci'on \textbf{ZCARGA\_CLIENTES\_SD}, la cual ejecuta el programa creado.
	Con la configuraci'on del M'odulo de Ventas y Distribuci'on, \textbf{SSA Beverage} podr'a llevar a cabo la automatizaci'on de sus operaciones comerciales, ya que la herramienta le permite tener un mayor control sobre el proceso de ventas y distribuci'on de sus productos. Adicionalmente, como SAP es un sistema integrado por otros m'odulos adicionales al de Ventas y Distribuci'on, no s'olo podr'a manejar el proceso comercial, sino tambi'en el proceso de elaboraci'on y almacenaje de los productos que ofrece, control de la calidad de los mismos, etc.
	Para finalizar, se sugieren las siguientes recomendaciones a seguir para el mantenimiento de 'este m'odulo:

\begin{itemize}
\item Tener reuniones frecuentes con personal de la empresa para gesti'on de nuevos requerimientos
\item Contar con un equipo de consultores para la captaci'on de dichos requerimientos, y para realizar las nuevas configuraciones (en caso de ser necesario)
\item Contar con un equipo de desarrolladores para aquellos requerimientos no configurables
\item Realizar cursos de capacitaci'on al personal de la empresa para la utilizaci'on del nuevo sistema
\end{itemize}