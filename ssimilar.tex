% Marco Teorico.
\chapter{Marco te'orico} \label{chap:ssimilar}

AQU\'I VA EL CONTENIDO DEL MARCO TE\'ORICO.
%Puedes quitar esto(es opcional)
\vspace{5 mm}

Los conceptos claves acerca de lo que es un sistema ERP, SAP, ABAP, Ascendant SAP son los que se presentan en este capítulo.

La secci'on \ref{sect:erp} presenta las nociones b'asicas acerca de en qu'e consiste un Sistema de Informaci'on ERP. La secci'on 
\ref{sect:sap} describe un poco lo que es SAP: sus caracter'isticas principales, una  breve historia. Por otro lado, la secci'on \ref{sect:abap} explica en que consiste el lenguaje de programaci'on ABAP, sus principales caracter'isticas. La 'ultima secci'on define la metodolog'ia \textbf{Ascendant SAP (ASAP)}, a ser utilizada a lo largo de este proyecto.

\section{Sistemas de Informaci'on ERP} \label{sect:erp}

Para entender en qu'e consisten los Sistemas de Informaci'on ERP, para ello habr'a que definir dos conceptos fundamentales: Sistema de Informaci'on, y m'as a'un, Sistemas ERP. 

\subsection{Definiciones} \label{subsect:defprop}

\begin{definicion} \label{def:sisinfo}

\end{definicion}

\begin{definicion} \label{def:siserp}
Un ERP (Sistema de Planificación de Recursos Empresariales, 'o como se le conoce en Ingl'es: \textbf{Enterprise Resource Planning}), es un sistema que tiene la capacidad de la automatizaci'on e integraci'on de todos los m'odulos de un 'area de negocio. En otras palabras, es capaz de manejar todas las 'areas relacionadas con una empresa de forma automatizada e integrada. 

Una caracter'istica fundamental de este tipo de sistemas es que est'a basado en m'odulos. Como consecuencia de ello, el mismo est'a compuesto por un conjunto de softwares o m'odulos. 

Para poder tener acceso a los datos que est'an relacionados con cada m'odulo es necesario tener una Base de Datos centralizada, la cual almacena dicha informaci'on. 
\end{definicion}

\section{SAP}

En esta secci'on se ofrece un panorama general acerca de SAP, qu'e es, cu'ales son sus caracter'isticas principales y cu'ales son los m'odulos que lo integra.

\subsection{Definici'on} \label{subsect:defprop}
\textbf{SAP AG} es una empresa de la rama de la Computaci'on que fue fundada por 4 ingenieros pertenecientes a IBM (\textbf{International Business Machines}), en la ciudad de Walldorf, Alemania, en el a~no de 1972. Por su origen alem'an, las siglas SAP son un acr'onimo de: \textit{Systeme, Anwendungen Produkte in der Datenverarbeitung}, que traducido al castellano significa: \textbf{Sistemas, Aplicaciones y Productos en Procesamiento de Datos}. 

El Software principal desarrollado por esta empresa es \textbf{\textit{SAP R/3}} y el cual se encuentra disponible en 28 idiomas. 'Este software es personalizable, utiliza la arquitectura cliente-servidor, es decir, que el cliente envia solicitudes al servidor y 'este a su vez envia una respuesta al cliente. 'Este software fue hecho en el lenguaje de programaci'on \textbf{ABAP/4}. 

\subsection{}










