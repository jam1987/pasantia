% Marco Teorico.
\chapter{Marco te'orico} \label{chap:ssimilar}



Los conceptos claves acerca de lo que es un sistema ERP, SAP, ABAP son los que se presentan en este capítulo.
\newline
\newline
La secci'on ~\ref{sect:erp} presenta las nociones b'asicas acerca de en qu'e consiste un Sistema de Informaci'on ERP. La secci'on 
~\ref{sect:sap} describe un poco lo que es SAP: sus caracter'isticas principales, una  breve historia. Por otro lado, la secci'on ~\ref{sect:abap} explica en que consiste el lenguaje de programaci'on ABAP, sus principales caracter'isticas. 

\section{Sistemas de Informaci'on ERP} \label{sect:erp}

Para entender en qu'e consisten los Sistemas de Informaci'on ERP, para ello habr'a que definir dos conceptos fundamentales: Sistema de Informaci'on, y m'as a'un, Sistemas ERP. 

\subsection{Definiciones} \label{subsect:defprop}

\begin{definicion} \label{def:sisinfo}
	Un Sistema de Informaci'on es un conjunto de componentes capaces de interactuar mutuamente con el objetivo de brindar apoyo a las actividades de una empresa o negocio.

\end{definicion}

\begin{definicion} \label{def:siserp}
Un ERP (Sistema de Planificación de Recursos Empresariales, 'o como se le conoce en Ingl'es: \textbf{Enterprise Resource Planning}), es un sistema que tiene la capacidad de la automatizaci'on e integraci'on de todos los m'odulos de un 'area de negocio. En otras palabras, es capaz de manejar todas las 'areas relacionadas con una empresa de forma automatizada e integrada. 
\newline
\newline
Una caracter'istica fundamental de este tipo de sistemas es que est'a basado en m'odulos. Como consecuencia de ello, el mismo est'a compuesto por un conjunto de softwares o m'odulos. 
\newline
\newline
Para poder tener acceso a los datos que est'an relacionados con cada m'odulo es necesario tener una Base de Datos centralizada, la cual almacena dicha informaci'on. 
\end{definicion}

\section{SAP} \label{sect:sap}

En esta secci'on se ofrece un panorama general acerca de SAP, qu'e es, cu'ales son sus caracter'isticas principales y cu'ales son los m'odulos que lo integra.

\subsection{Definici'on} \label{subsect:defprop}
\textbf{SAP AG} es una empresa de la rama de la Computaci'on que fue fundada por 4 ingenierosque pertenecieron a IBM (\textbf{International Business Machines}), en la ciudad de Walldorf, Alemania, en el a~no de 1972. Por su origen alem'an, las siglas SAP son un acr'onimo de: \textit{Systeme, Anwendungen Produkte in der Datenverarbeitung}, que traducido al castellano significa: \textbf{Sistemas, Aplicaciones y Productos en Procesamiento de Datos}. 
\newline
\newline
El Software principal desarrollado por esta empresa es \textbf{\textit{SAP R/3}} y el cual se encuentra disponible en 28 idiomas. 'Este software es personalizable, utiliza la arquitectura cliente-servidor, es decir, que el cliente envia solicitudes al servidor y 'este a su vez envia una respuesta al cliente. 'Este software fue hecho en el lenguaje de programaci'on \textbf{ABAP/4}. 

\subsection{Caracter'isticas de SAP R/3}
De acuerdo al autor de \cite{SAP01}, las caracter'isticas del software \textbf{SAP R/3} se pueden dividir en diferentes categor'ias, las cuales se mencionan a continuaci'on:

\subsubsection*{Caracter'isticas Generales}
\begin{itemize}
\item Es un software altamente integrado y multifuncional, lo que trae como consecuencia que exista una estrecha relaci'on entre las funciones del mismo.
\item Es una aplicaci'on que trabaja en tiempo real. En otras palabras, las actualizaciones de los datos son efectuadas a trav'es de una conexi'on, y en ese mismo instante.
\end{itemize}
\subsubsection*{Caracter'isticas de Negocio}
\begin{itemize}
\item Este software contiene todas las funcionalidades necesarias para poder llevar a cabo el manejo de un negocio entero. 'Este incorpora una aplicaci'on llamada \textit{Best Industry Practices}, que traducido al espa\~nol quiere decir: \textit{Mejores Practicas de la Industria}, y 'este 'ultimo es adecuado para una amplia gama de industrias y organizaciones.
\item Este programa es capaz de soportar todos los procesos de negocio de la empresa.
\end{itemize}

\subsubsection*{Caracter'isticas de Flexibilidad}
\begin{itemize}
\item Este software es altamente configurable. En otras palabras, se puede adaptar a las necesidades de la empresa que lo utilice y a sus requerimientos. Para ello, se pueden realizar cambios que, dependiendo del n'umero de factores que participen, 'estos tendr'an su grado de complejidad.
\item Es capaz de dar apoyo a empresas que poseen subsidiarias en distintas partes del mundo.
\item Este software es muy utilizado a nivel mundial, dado que est'a disponible en 28 idiomas, y a que posee la capacidad de adaptarse a la moneda, leyes y regulaciones, impuestos para un cierto pa'is, etc.
\end{itemize}

\subsubsection*{Caracter'isticas T'ecnicas}
\begin{itemize}
\item Este software tiene la capacidad de ser portable, dado que es multiplataforma, es decir, soporta cualquier sistema operativo, manejador de Base de Datos, etc.
\item Posee un n'umero m'inimo de redundancia, lo que favorece a la consistencia de los datos almacenados. Adicionalmente, posee un manejador de alta seguridad de los datos, y puede manejar etructuras de datos complejas.
\end{itemize}

\subsubsection*{Otras Caracter'isticas}
\begin{itemize}
\item Tiene la capacidad de manejar la misma informaci'on en cada m'odulo.
\item Posee una 'unica manera de ingreso al sistema. 'Este es a trav'es del \textit{SAP GUI}.
\item Tiene la capacidad de ser escalable, es decir, que est'a preparado para manejar el continuo crecimiento del trabajo sin disminuir la calidad.
\item Tiene una interfaz gr'afica amigable.
\end{itemize}

\subsection{Adaptaci'on del Software a las Empresas}
Para poder realizar la adaptaci'on de SAP a las necesidades de una empresa en particular, est'an un conjunto de herramientas y utilidades destinado a ello. Para esto, se requiere de un conjunto de consultores, un equipo de proyecto y personal de Tecnolog'ia de la Informaci'on (IT), quienes ser'an los encargados de efectuar dicha adaptaci'on. 
\newline
\newline
Este proceso se puede realizar a trav'es de dos m'etodos, los cuales se listan a continuaci'on:

\begin{itemize}
\item \textbf{Cambios en la Configuraci'on:} Aqu'i son modificadas las tablas relacionadas con los distintos m'odulos para poder realizar la adaptaci'on.
\item \textbf{Programaci'on en el Lenguaje ABAP/4:} Esto implica modificar programas ya existentes en \textbf{SAP R/3} o crear programas nuevos.
\end{itemize}

\subsection{M'odulos de SAP R/3}
Debido a que SAP es un ERP, luego, SAP est'a dividido en diferentes m'odulos, con el fin de poder abarcar cada 'area de una empresa. Estos m'odulos son los que se listan a continuaci'on:

\begin{enumerate}
\item Asset Management (Manejo de Aplicaciones - AM)
\item Financials (Finanzas - FI)
\item Controlling (Control - CO)
\item Human Resources (Recursos Humanos - HR)
\item Plant Maintenance (Mantenimiento de Planta - PM)
\item Production Planning (Planificaci'on de la Producci'on - PP)
\item Project System (Sistema de Proyectos - PS)
\item Quality Management (Manejo de Calidad -QM)
\item Sales and Distribution (Ventas y Distribuci'on - SD)
\item Materials Management (Manejo de Materiales - MM)
\item Services Management (Manejo de Servicios - SM)
\item Industry Specific Solutions (Soluciones Espec'ificas para la Industria - IS)
\item Business Workflow (Flujo de trabajo del Negocio - WF)
\item Basis (Incluye el lenguaje de programaci'on \textbf{ABAP 4} - BC).

\end{enumerate}

\section{M'odulo de Ventas y Distribuci'on (SD)} \label{sect:sd}

El \textbf{M'odulo de Ventas y Distribuci'on} (Sales and Distribution como es conocido en ingl'es) es un sub-sistema perteneciente a SAP, el cual se encarga de prestar apoyo a las distintas empresas en el 'area de las Ventas y distribuci'on de productos y/o servicios. 
\newline
\newline
Este m'odulo ayuda a las compa\~n'ias a establecer un precio para sus productos, chequear 'ordenes de ventas que se mantienen abiertas, a tomar previsiones para necesidades futuras, etc. Adicionalmente, ayuda a dar mayor control a las actividades relacionadas con el 'area de ventas: desde el momento en que ocurre un pedido de alg'un(algunos) producto(s) y/o servicio(s) hasta su posterior entrega.



\subsection{Herramientas Principales}
\subsubsection{Manejo de Precios e Impuestos}
A trav'es de esta herramienta, el m'odulo puede evaluar los precios que son colocados a los productos o servicios de acuerdo a unas condiciones establecidas previamente. 
\subsubsection{Chequeo de Disponibilidad}
Con esta herramienta, el m'odulo puede evaluar la disponibilidad de un producto para un almac'en especificado.
\subsubsection{Manejo de Cr'edito}
Con esta herramienta es posible establecer l'imites de cr'edito para un cliente durante el proceso de ventas en el cual se encuentra envuelto.
\subsubsection{Facturaci'on}
Una vez que una Orden de Ventas es creada, se utiliza esta herramienta para crear la(s) factura(s) asociadas.
\subsubsection{Determinaci'on del Material}
Con esta herramienta es posible determinar un material específico de acuerdo a unas condiciones especificadas.
\subsubsection{Determinaci'on de Cuentas}
Ayuda a obtener ciertos detalles de los clientes bas'andose en unas condiciones espec'ificas.
\subsubsection{Procesamiento de Textos}
Con esta herramienta se hace posible el manejo de textos entre los distintos documentos que se obtienen del proceso de ventas.


\subsection{Clasificaci'on de los Datos en el M'odulo SD}
De acuerdo al autor de \cite{SD01}, la data almacenada dentro del M'odulo se puede clasificar como se muestra en la figura ~\ref{fig:datasd} donde se muestra la siguiente clasificaci'on:.
\subsubsection*{Datos Maestros}
Los Datos Maestros dentro del M'odulo SD est'an compuestos por:
\begin{itemize}
\item Datos de Compa\~n'ias
\item Datos Maestros de Clientes
\end{itemize}
Cada una de estas entidades contienen a su vez atributos, jerarqu'ias y tablas.

\subsection{Proceso de Ventas utilizando el M'odulo SD}
	En este m'odulo los dos objetos m'as importantes son: las clases de documentos y las condiciones. El primero, porque en cada fase del proceso se elabora un documento que contiene informaci'on relevante para dicha fase. La segunda, porque contiene las cl'ausulas por las cuales se va a regir el esquema de precios.
	En la Figura ~\ref{fig:salesFlow} se muestra el proceso general de Ventas y Distribuci'on que es cubierto por SAP. A continuaci'on, se proceder'a a explicar cada fase mostrada en dicha imagen.

\subsubsection*{Solicitud de Informaci'on de Productos}
	Este es el punto de partida para el proceso de ventas. En este paso, el cliente solicita informaci'on acerca de los productos y/o servicios que son ofrecidos por la empresa, con el objetivo de una posible adquisici'on de alg'un producto y/o servicio.
	
\subsubsection*{Creaci'on de Orden de Ventas}
	En este punto, el cliente le solicita a la empresa un pedido de un material determinado solicitado en el paso anterior. Para esto, se crea en SAP lo que se conoce como un \textbf{Pedido de Venta}. Un \textbf{Pedido de Venta} es un documento que contiene informaci'on acerca del(los) material(es) que se est'a solicitando, entre otras cosas. Un Pedido tiene la siguiente divisi'on:
\begin{itemize}
\item \textbf{Cabecera del Documento:} En esta parte del documento se recoge informaci'on acerca de la informaci'on general del pedido. Entre la informaci'on relevante que se puede encontrar, se pueden mencionar: 
\begin{enumerate}
\item Fecha del Pedido
\item Cliente que hace el Pedido
\item Cliente que recibe el Pedido
\item Tipo de Pedido (Si es un Pedido est'andar, Contrato con el cliente, solicitud de Nota de Cr'edito/D'ebito, Retorno de Productos, Consulta de Cliente, solicitud de Presupuesto)
\item N'umero de Pedido: El el n'umero con el cual el cliente identifica su pedido.
\item N'umero de Documento: Es el n'umero con el cual se identifica un'ivocamente el Pedido de Ventas.
\end{enumerate}
\item \textbf{Posiciones del documento:} Contiene informaci'on sobre los materiales solicitados. Cada material va colocado en una posici'on diferente, y en 'esta se reflejan los siguientes datos:
\begin{enumerate}
\item N'umero de Material: Es el n'umero que identifica un'ivocamente al material solicitado.
\item Descripci'on del Material: Es un nombre que se le coloca al material.
\item Cantidad
\item Unidad de Venta: Es la unidad en la cual se vende el material. Esto ocurre porque dentro de la informaci'on que posee el material, se diferencian dos unidades: la de almacenamiento y la de venta
\item Precio bruto: Es el precio del material.
\end{enumerate}
\end{itemize}

Para esto se tienen las siguientes transacciones, las cuales son las m'as utilizadas:
\begin{itemize}
\item VA01: Creaci'on de Pedidos
\item VA02 Modificaci'on de Pedidos
\item VA03: Visualizaci'on de Pedidos
\end{itemize}
	
\subsubsection*{Entrega de Bienes y/o Servicios}
	El segundo paso a ejecutar una vez que se cre'o el pedido de ventas, es la creaci'on de una  Entrega. Para esto, se crea un nuevo documento el cual es llamado \textbf{Documento de Entrega}. Esto es, porque en el mencionado documento, se detalla la informaci'on relacionada con el proceso de entrega y transporte. En este documento, adem'as de la informaci'on que es copiada del documento anterior (Cabecera y Posiciones), se detalla informaci'on acerca de las cantidades reales entregadas, ya que las cantidades solicitadas en el Pedido est'an sujetas a la disponibilidad de las mismas en los Almacenes. Para ello, es necesario el uso de las siguientes transacciones:
\begin{itemize}
\item VL01N: Creaci'on de Entregas
\item VL02N: Modificaci'on de Entregas
\item VL03N: Visualizaci'on de Entregas
\end{itemize}
\indent Una vez que la entrega es creada, se debe contabilizar el material, para que la entrega est'e considerada como completada. Esto es, para que pueda procederse al siguiente paso.
	
\subsubsection*{Facturaci'on}
	La siguiente acci'on a realizar dentro del ciclo es la facturaci'on. En este paso, como su nombre lo indica, se crea la factura relacionada con un pedido previamente realizado. En este documento, se reflejan los materiales y/o servicios solicitados, con sus respectivas cantidades y su monto. Adicionalmente, es reflejado los impuestos que apliquen, asi como  el monto total a pagar por el solicitante. Para esto, es necesario el uso de las siguientes transacciones:
\begin{itemize}
\item VF01: Creaci'on de Facturas
\item VF02: Modificaci'on de Facturas
\item VF03: Visualizaci'on de Facturas
\end{itemize}
\indent Es posible, en primer lugar, la creaci'on de varias facturas. Esto es posible gracias a las negociaciones que lleva la empresa con el cliente, sobre los montos a pagar. En segunda instancia, si hubo errores en la factura creada, es posible la anulaci'on de la misma, con lo cual la deuda adquirida por el cliente se anula, en caso de no tener otra factura pendiente. Con esto es posible volver a crear la factura con las correcciones a aplicar. 
\newline
\newline
\indent Una vez creada la factura, se debe contabilizar la misma. Esto es con el fin de que se cree un documento contable, para que la nueva venta est'e reflejada dentro de la contabilidad de la empresa.
	
\subsubsection*{Pago por los bienes y/o servicios adquiridos}
	Este es el paso final del ciclo de ventas que es llevado en el m'odulo SD. Para ello, de acuerdo a los planes de pago establecidos entre la empresa y el cliente, la misma inicia el cobro de la deuda adquirida.
	

\subsection{Relaci'on existente entre el M'odulo SD y otros M'odulos}
'Este m'odulo esta fuertemente integrado con otros m'odulos de SAP, como por ejemplo: MM, WM, QM. 
\newline
\newline
En el momento en que un cliente realiza un pedido de alg'un producto, posteriormente se chequea la disponibilidad del mismo en alg'un almac'en; esto es posible gracias al m'odulo MM. 
\newline
\newline
Por otro lado, el m'odulo QM es el encargado de manejar la calidad y brindar soporte a un servicio prestado al cliente, ambos representados por un documento de ventas en SD. 

\section{Lenguaje de Programaci'on ABAP/4} \label{sect:abap}
ABAP (Advanced Business Application Programming - Programaccion de Aplicaciones de Negocio Avanzado) es un lenguaje de programaci'on que fue dise\~nado en la d'ecada de los 80. Su uso principal es la de generar reportes con los cuales se les permite a las empresas construir sus propias aplicaciones para el manejo de las distintas 'areas que lo componen (manejo de materiales, manejo del 'area financiera, manejo de las ventas, etc).
\newline
\newline
Este es uno de los primeros lenguajes de programación que incluye dentro de su definición el concepto de Bases de Datos L'ogicas. 
\newline
\newline
Dentro de las caracter'isticas que posee el lenguaje, se pueden mencionar las siguientes:
\begin{itemize}
\item Es un lenguaje basado en la programaci'on estructurada. En otras palabras, contiene estructuras de control.
\item Es un lenguaje interpretado, aunque existen versiones compiladas del mismo.
\item Es muy utilizado para obtener dos tipos de programas: los que son usados para obtener por ejemplo un listado (modo reporte), y aquellos que son usados como transacciones (modo di'alogo).
\item Es un lenguaje orientado a eventos, es decir, que puede ser controlado desde el exterior a trav'es de sentencias de eventos.
\item Est'a integrado totalmente con el sistema \textbf{SAP R/3}.
\item La salida de sus programas es multilingual. 
\end{itemize}

\subsection{Estructura de un programa ABAP/4}
	Un programa ABAP tiene la responsabilidad del procesamiento de datos dentro de una aplicaci'on. Esto quiere decir que el programa debe ser dividido en distintas secciones que pueden ser asignadas a cada paso secuencial a ejecutar dentro del programa. Para poder lograr esto, es necesario modularizar los programas hechos en ABAP. Cada m'odulo o \textbf{Bloque de Procesamiento} como tambi'en se le conoce, consiste en un conjunto de instrucciones ABAP. Al momento de ejecutar un programa, es posible realizar la llamada de cada bloque. Es importante resaltar que no permiten anidamiento.
\newline
\newline
	Cada programa ABAP est'a compuesto por dos partes principales, las cuales se explican en el Ap'endice ~\ref{sect:programabap}.

\subsection{Includes en ABAP/4}
	Un Include es un programa especial el cual se hace en SAP con el objetivo de lograr mayor modularizaci'on del programa principal.
\newline
\newline
	Este tiene dos funciones principales:
\begin{itemize}
\item Librer'ia: Los programas Include permiten el uso del mismo c'odigo fuente en diferentes programas. Por ejemplo, en los Includes se pueden realizar las declaraciones globales, la definici'on de clases y procedimientos, etc.
\item Orden: Los programas Include permiten manejar un cierto orden dentro de programas complejos. Por ejemplo, los grupos de funciones usan los programas Include para almacenar partes del programa. 
\end{itemize}
	

\subsection{Herramientas provistas por ABAP/4}
	Dentro de las herramientas m'as utilizadas de \textbf{ABAP/4}, est'an las que se listan a continuaci'on:
\begin{itemize}
\item Smartforms (Formularios)
\item Sesiones de Batch Input (Carga masiva de datos)
\end{itemize} 
	En la siguientes subsecciones se explicar'a m'as a detalle acerca de dichas herramientas.
	
\subsubsection{Smartforms}
	\textbf{Los Smartforms} fueron introducidos dentro del lenguaje a trav'es de la versi'on 4.6 de \textbf{SAP R/3}. Su funci'on principal es la impresi'on y env'io de documentos a trav'es del correo electr'onico, o por fax. 
\newline
\newline
	Con esta herramienta es posible hacer el dise\~no de formularios, archivos PDF y documentos en general. 
	Esta herramienta provee una interfaz capaz de construir y mantener la disposici'on y l'ogica del formulario.
	Una ventaja que tiene esta herramienta es el uso  de una interfaz gr'afica para modificar formularios, ya que con esto, se evita el uso de la programaci'on.
\newline 
\newline
	En un smartform, los datos son transmitidos a trav'es de t'ablas din'amicas o est'aticas. Adicionalmente permite incluir gr'aficos que pueden ser visibles en el formulario. En los pr'oximos segmentos se explicar'a los componentes de esta herramienta.
	
\subsection{Sesi'on de Batch Input (Carga Masiva de Datos)}
	Esta es una herramienta que provee ABAP/4 con el fin de introducir datos de manera no interactiva dentro del Sistema SAP. La Carga Masiva o por Lotes (Batch Input) es usuado muy frecuentemente para transferir datos desde sistemas externos a un sistema SAP, o entre sistemas SAP. Una sesi'on de Batch Input, m'as espec'ificamente, es un conjunto de una o varias llamadas a transacciones con los datos a ser procesados por dichas transacciones.  
\newline
\newline
	Para poder lograr esta tarea, existen dos transacciones en SAP: la SHDB y la SM35. A partir de la ejecuci'on de alguna de las transacciones antes mencionadas, el sistema se encarga de realizar una grabaci'on sobre la(s) transacción(es) involucrada(s) junto con los datos en un formato especial el cuan puede ser interpretado por SAP. Cuando el sistema ejecuta una sesi'on, utiliza la data almacenada en dicha sesi'on para comenzar la simulaci'on de la entrada on-line de los datos. El sistema llama a las transacciones y carga los datos en ella. En la Figura~\ref{fig:process}  se puede apreciar los pasos a seguir para poder llevar a cabo una sesi'on de Batch Input. Para ello, el sistema a trav'es de una interfaz recibe el archivo que contiene los datos a ser cargados, y 'este es pasado al programa de Carga Masiva (Batch Input), el cual se encarga de procesar dichos datos y colocarlos en una cola, para luego procesar uno por uno a trav'es de la simulaci'on de la ejecuci'on de la transacci'on involucrada. 

