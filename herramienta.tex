% Planteamiento del Problema
\chapter{Marco Metodol'ogico} \label{chap:metodologia}
Para el desarrollo de un buen software es necesario la utilizaci'on de una metodolog'ia, ya que la misma brinda una serie de mecanismos y fases para el desarrollo organizado de la aplicaci'on en cuesti'on.
Para este proyecto, se decidi'o utilizar la metodolog'ia \textbf{ASAP (Ascendant SAP)}, ya que es la m'as utilizada para el desarrollo de aplicaciones en SAP ERP.
\section{Descripci'on de la metodolog'ia}
\textbf{ASAP (Ascendant SAP 'o Accelerated SAP como tambi'en se le conoce)} es una metodolog'ia que fue dise\~nada por SAP para agilizar el desarrollo de sus aplicaciones, ya que con la metodolog'ia actual, un desarrollo se pod'ia llevar mucho m'as tiempo del esperado. Mientras un proyecto usando la metolodog'ia convencional se puede llevar dos a\~nos o m'as en realizarse, el mismo proyecto bas'andose en la metodolog'ia \textbf{ASAP}, puede ser realizado en menos de un a\~no. 
Es importante resaltar que esta tecnolog'ia no es utilizada en todos los desarrollos con SAP, ya que la misma se recomienda para aquellas empresas que no tienen una extensa modificaci'on en sus requerimientos, o para aquellas empresas que requieren del uso de la re-ingenier'ia.
\section{Caracter'isticas Principales}
ASAP ha sido dise\~nado con el objetivo de estandarizar y de llevar de una forma coordinada una implementaci'on SAP.  Esta metodolog'ia posee las siguientes caracter'isticas:
\begin{enumerate}
\item Es capaz de optimizar tiempo, calidad y recursos.
\item Es capaz de aprovechar las mejores pr'acticas del negocio.
\item Es capaz de entregar un proceso orientado a un mapa de proyecto (Hoja de Ruta ASAP). Una hoja de ruta ASAP es un gr'afico que presenta los pasos o fases a seguir, como se muestra en el siguiente ejemplo: 

\end{enumerate}

\section{Fases de la metodolog'ia ASAP}
Esta metodolog'ia est'a dividida en 4 fases, las cu'ales se listan en la siguiente secci'on.
\subsection{Fase 1: Preparaci'on del Proyecto}
En esta fase se evalúan dos factores cr'iticos del proyecto: en primer lugar la preparaci'on de la organizaci'on del mismo, ya que en este punto se realizan algunas tareas de gesti'on que son claves, como por ejemplo: Proveer un compromiso de la alta gerencia y apoyo, establecer metas claras y objetivos claros, acordar en los próximos pasos a dar dentro del proyecto, proveer un proceso eficiente de toma de decisiones, escoger un equipo que sea calificado y que represente las distintas 'areas funcionales.
En segundo lugar, la planificaci'on del proyecto, ya que en este punto se deben identificar aquellos elementos que sean cr'iticos, dentro de los cuales se puede mencionar los que se listan a continuaci'on:
\begin{itemize}
\item Principios Rectores: Estos son principios de alto nivel que pueden ser establecidos al inicio del proyecto. Estos definen y comunican la visi'on de la empresa, ayuda a mantener el proyecto enfocado, y en el caso en que exista alg'un conflicto, sirve de base para su soluci'on. 
\item Principios estrat'egicos:  Estos son principios de negocio que direccionan las estrategias a utilizar. Al tener unas estrategias bien definidas, la implementaci'on se torna mas f'acil, y as'i se pueden lograr los objetivos establecidos.
\item Impulsores del proyecto: Estos son los encargados de escoger el software ERP para una implementaci'on particular. Es importante recordar que SAP tiene distintas soluciones para ciertos tipos de empresas, como por ejemplo IS-OIL para aquellas empresas cuyo producto tenga relaci'on con hidrocarburos, entre otros. 
\item Presupuestos, est'andares y indicadores: Al inicio del proyecto se debe establecer un presupuesto sobre el costo del mismo, adem'as que se deben definir los est'andares que va a utilizar y los distintos indicadores.
\end{itemize}
En tercer lugar, el equipo de implementaci'on, ya que es el responsable de que el proyecto se pueda llevar a cabo. Generalmente este equipo se encuentra integrado por consultores pertenecientes a organizaciones externas y por empleados internos de la compa\~n'ia. Este equipo est'a dividido en los siguientes grupos:
\begin{itemize}
\item Personal interno de la compa\~n'ia o Cliente.
\item Personal de Implementaci'on e Integraci'on: 'Este incluye a empleados de la empresa desarrolladora o de sub-contratistas.
\end{itemize}
El equipo de clientes est'a integrado por:
\begin{itemize}
\item Miembros del equipo central: Tienen dedicaci'on al 100 del tiempo disponible dedicado al proyecto.
\item Equipo de extensi'on: Son los que tienen dedicaci'on del 20-50 del tiempo al proyecto, dependiendo de la fase en la cual se encuentre.
\end{itemize}
En este punto, los equipos definidos son organizados por los distintos m'odulos o funcionalidades. Por ejemplo, se pueden tener equipos para el m'odulo de Finanzas (FI), Ventas y Distribuci'on (SD), Gesti'on de Materiales (MM), entre otros. 
Es fundamental que el equipo escogido tenga las siguientes capacidades:
\begin{itemize}
\item Analizar el impacto del nuevo sistema ERP sobre los procesos del negocio contra el proceso actual.
\item Analizar los requerimientos funcionales y de implementaci'on. 
\item Dise\~nar un sistema integrado.
\item Proveer de conocimiento a los empleados durante el proyecto.
\end{itemize}
\subsection{Fase 2: Business Blueprint}
El objetivo principal de esta fase es comprender el funcionamiento actual de la empresa, para as'i poder determinar los requerimientos de implementaci'on basados en las necesidades que la organizaci'on pueda presentar en un futuro. Para esto, se realiza un an'alisis exhaustivo del negocio de la compa\~n'ia, c'omo se desenvuelve actualmente, e identificando las funcionalidades soportadas por el sistema actual. Luego se compara las pr'acticas existentes y sus funcionalidades  con las que son soportadas por SAP. 
Durante esta fase, los ejecutivos y gerentes de la compa\~n'ia son entrevistados. Dichas entrevistas son realizadas dentro de grupos peque\~nos, o de manera individual. Luego, basado en las respuestas obtenidas, los consultores pueden entender y definir los siguientes par'ametros:
\begin{itemize}
\item El negocio de la compa\~nia
\item La forma de operar
\item Elementos cr'iticos del negocio
\item Procesos deseables a llevar a cabo en el negocio
\item Requerimientos del negocio y de funcionalidades
\item El alcance que tendr'a el proyecto
\item Los riesgos que posee el desarrollo del proyecto
\end{itemize}
Al final de esta etapa es elaborado un documento llamado \textbf{Documento de Proyecto (Blueprint Document como se le conoce en ingl'es)}. 
Este documento puede ser descrito como un modelo visual de la empresa. En este documento se detalla lo siguiente:
\begin{itemize}
\item Funcionalidad ya existente
\item Funcionalidad a desarrollar
\item Procesos actualmente en operacion
\item Alcance de la implementaci'on
\item Estructura organizacional
\item Funcionalidad diferida
\item Riesgos potenciales
\item
\end{itemize}
Una decisi'on fundamental que se toma en esta fase es la definici'on de una estructura organizacional SAP basada en los procesos de organizaci'on del negocio. 
	Esta estructura determina como los datos son definidos dentro del sistema, la complejidad de los datos de entrada, y el tama\~no de los archivos que contienen datos maestros. 
	La estructura definida debe haberse analizado bien en esta etapa, ya que cualquier cambio que fuera realizado en las fases siguientes, derivar'ia en un costo elevado.
	Uno de los elementos m'as importantes de la estructura organizacional es el c'odigo que se le asigna a la compa\~n'ia, ya que 'este es el elemento m'as alto dentro de esta estructura armada. El c'odigo de la compa\~n'ia una uniad legal e organizacionalmente independiente. 'Este representa a una unidad de contabilidad independiente, lo cual hace que posea sus propias componentes financieras.
	Una de las partes de esta estructura es el \textbf{'Area de Control}. La misma simboliza a un elemento organizacional con la cual se manejan como estructura organizativa es la estructura del negocio. Esto no es m'as que la organizaci'on de la empresa en s'i. En otras palabras, las distintas 'areas funcionales, como Log'istica, Recursos Humanos, etc.
\subsection{Fase 3: Realizaci'on}
	Durante esta fase, el sistema es configurado basado en los requerimientos obtenidos en la fase anterior, y luego es probado. Esta fase no es r'igida, es decir, que la aplicaci'on de la misma es progresiva. En otras palabras, se construye, se prueba, se refinan los detalles y se vuelve a probar. En las pr'oximas sub-secciones se describen un poco las etapas por las cuales se pasa dentro de esta fase.

\subsubsection{Simulaci'on}
	El primer paso que se lleva en esta etapa es la configuraci'on. Aqu'i, los distintos consultores configuran el sistema SAP de acuerdo al documento del Negocio previamente definido.  Esta configuraci'on cubre el 80 de los procesos del negocio de la compa\~n'ia y de las transacciones diarias. Este proceso implica modificar el software de SAP a trav'es de procedimientos no programables, como por ejemplo:  modificaciones de entrada a tabla, herramientas base, etc.
	El siguiente paso dentro de esta etapa es la Reproducci'on. 'Este consiste en introducir un conjunto reducido de usuarios en el nuevo sistema para hacer las pruebas a la configuraci'on realizada previamente, y as'i poder obtener una retroalimentaci'on. Las reproducciones son realizadas de forma peri'odica bas'andose en las preferencias de los usuarios. 
	
\subsubsection{Validaci'on}
	Durante esta etapa, el dise\~no es refinado y finalizado. El equipo de trabajo refina el sistema de tal modo que todos los requerimientos del negocio se encuentren configurados. 
	Un punto importante durante la realizaci'on de esta fase es la elaboraci'on de una lista de procesos maestros. 	El equipo involucrado comienza a desarrollar los procedimientos del procedimiento del Negocio (Gu'ia de Configuraci'on), con la cual se documenta toda la configuraci'on realizada al sistema. 
\subsubsection{Uni'on y Pruebas de Integraci'on}
	Para que un software sea colocado en producci'on debe haber recibido todas las pruebas necesarias. Un software que no es probado de la manera adecuada, puede traer consecuencias a la hora de ponerse en reproducci'on, como por ejemplo que se presente alguna falla inesperada, lo que derivar'ia en descartar el programa.
	El hecho de realizar una prueba a un sistema ERP integrado representa un gran desaf'io, pero dentro de los beneficios a obtener est'an los siguientes: 
\begin{itemize}
\item Se confirma que el proceso trabaja adecuadamente
\item Se obtiene una configuraci'on 'agil
\item Tiene rendimiento garantizado
\item La integraci'on es mejorada
\item Sus costos son bajos
\item Los riesgos se reducen al m'inimo
\end{itemize}
\subsection{Fase 4: Preparaci'on Final}

AQU\'I VA EL DESARROLLO DEL PLANTEAMIENTO DEL PROBLEMA.
%Puedes quitar esto(es opcional)
\vspace{5 mm}

Para verificar si los m'etodos para detecci'on de ataques de denegaci'on de
servicio mediante el uso del par'ametro de Hurst, descritos en la secci'on
son utilizables en tiempo real necesitamos implementarlos e incluirlos en 
una herramienta que adem'as maneje las trazas y sea capaz de graficar los resultados. 


Los requerimientos funcionales y no funcionales para la creaci'on del programa,
son descritos en las secci'on \ref{sect:requirements}. 

\section{Requerimientos} \label{sect:requirements}

Los requerimientos de la herramienta o software a implementar se mencionan a
continuaci'on: 

\begin{itemize}
\item Por posibles cuestiones legales, la implementaci'on de la soluci'on 
debe ser de c'odigo libre para poder, luego de su validaci'on, ser
modificados y los m'odulos e incluidos en el AIR-NMS del laboratorio Kinoshita
de la Universidad de Tohoku.
\item La herramienta de l'inea de comando debe desarrollarse en el lenguaje
{\tt C}, ya que se quiere utilizar la librer'ia {\tt libpcap} para manipular
las trazas de red como fuese necesario.
\item Con el uso de la liber'ia {\tt libpcap} se debe extraer todo la
informaci'on posible sobre los tiempos de llegada, cantidad y tama~no de los
paquetes de distintos protocolos que componen TCP/IP en la traza que alimenta
el programa.
\item Debido a que algunos m'etodos de estimaci'on del par'ametro de Hurst
utilizan gr'aficas, la herramienta debe tener capacidades gr'aficas. 
El software debe tambi'en poder graficar el cambio del par'ametro en el tiempo
cuando se usa el mecanismo de ventanas deslizantes, y la serie de tiempo
creada a partir de la velocidad de captura ($c$).
\item El programa debe ser flexible. Esto incluye el hacer todos los aspectos
importantes de la estimaci'on parametrizables, tales como la velocidad de 
captura ($c$), el tama~no de la ventana ($w$) y el tama~no de la ventana
deslizante ($s$).
\item El programa debe incluir al menos 3 m'etodos para la estimaci'on del 
par'ametro de Hurst. Dichos m'etodos fueron escogidos desde un principio por
el laboratorio Kinoshita y los criterios de selecci'on se exponen a
continuaci'on. Ellos ya ten'ian una implementaci'on del estad'istico R/S y
quer'ian seguir teniendo este m'etodo como alternativa. Debido a los art'iculos 
\cite{intelligentfuzzy} y \cite{xiang:292} d'onde se hace un an'alisis
exhaustivo de la estimaci'on del par'ametro de Hurst mediante las gr'aficas
varianza-tiempo y su comportamiento en ataques de denegaci'on de servicio era
tambi'en razonable su implementaci'on. Por 'ultimo, el m'etodo de la varianza
modificada de Allan fue seleccionado ya que hoy en d'ia es uno de los m'etodos 
m'as novedosos utilizados para la estimaci'on del par'ametro de Hurst y se ha
demostrado que es bastante preciso, especialmente con muestras peque~nas
\cite{MAVARStefano}. 
\item  Por tratarse de un prototipo, todas sus funciones ser'ian para el 
an'alisis de forma {\it offline}. Los resultados dar'ian una idea del
comportamiento en un ambiente real.
\end{itemize}


