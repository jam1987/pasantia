\setcounter{page}{4}
\begin{center}
	{\bf Resumen}
\end{center}	

AQU\'I VA EL CONTENIDO DEL RESUMEN.
%Puedes quitar esto(es opcional)
\vspace{5 mm}

Como parte de este sistema, se plante'o la evaluaci'on de un m'etodo de detecci'on de
ataques de denegaci'on de servicio mediante la estimaci'on del parametro de
Hurst utilizando el mecanismo de ventanas deslizantes. Los m'etodos se
incorporaron en una herramienta de l'inea de comando vers'atil capaz de obtener
series de tiempo apartir de una traza de datos tomados de la red y de graficar
las estimaciones del par'ametro de Hurst. La herramienta fue evaluada mediante
el uso del algoritmo de Paxson con respecto a su precisi'on y con trazas de los
escenarios de ataques de denegaci'on de servicio producidas en el laboratorio
Lincoln del {\it Massachusetts Institute of Technology}, durante la evaluaci'on
dirigida por la {\it Defense Avadanced Research Projects Agency}, con respecto
a su capacidad de detecci'on de ataques. Los resultados muestran que el m'etodo
de detecci'on se puede implementar para su uso en tiempo real y se puede
mejorar con respecto a la detecci'on de ataques.
\newpage

